\documentclass[11pt]{article}
\usepackage{amsmath,amssymb,amsthm}

\DeclareMathOperator*{\E}{\mathbb{E}}
\let\Pr\relax
\DeclareMathOperator*{\Pr}{\mathbb{P}}

\newcommand{\eps}{\varepsilon}
\newcommand{\inprod}[1]{\left\langle #1 \right\rangle}
\newcommand{\R}{\mathbb{R}}

\newcommand{\handout}[5]{
  \noindent
  \begin{center}
  \framebox{
    \vbox{
      \hbox to 5.78in { {\bf CS 224: Advanced Algorithms } \hfill #2 }
      \vspace{4mm}
      \hbox to 5.78in { {\Large \hfill #5  \hfill} }
      \vspace{2mm}
      \hbox to 5.78in { {\em #3 \hfill #4} }
    }
  }
  \end{center}
  \vspace*{4mm}
}

\newcommand{\lecture}[4]{\handout{#1}{#2}{#3}{Scribe: #4}{Lecture #1}}

\newtheorem{theorem}{Theorem}
\newtheorem{corollary}[theorem]{Corollary}
\newtheorem{lemma}[theorem]{Lemma}
\newtheorem{observation}[theorem]{Observation}
\newtheorem{proposition}[theorem]{Proposition}
\newtheorem{definition}[theorem]{Definition}
\newtheorem{claim}[theorem]{Claim}
\newtheorem{fact}[theorem]{Fact}
\newtheorem{assumption}[theorem]{Assumption}

% 1-inch margins, from fullpage.sty by H.Partl, Version 2, Dec. 15, 1988.
\topmargin 0pt
\advance \topmargin by -\headheight
\advance \topmargin by -\headsep
\textheight 8.9in
\oddsidemargin 0pt
\evensidemargin \oddsidemargin
\marginparwidth 0.5in
\textwidth 6.5in

\parindent 0in
\parskip 1.5ex

\begin{document}

\lecture{NUMBER --- DATE, 2013}{Fall 2013}{Prof.\ Jelani Nelson}{YOUR NAME}

\section{Overview}

In the last lecture we \ldots.

In this lecture we \ldots.

\section{Math things}
Here is an equation:
\begin{equation*}
a^2 + b^2 = c^2 ,
\end{equation*}
or
$$
a^2 + b^2 = c^2 .
$$

Here's the same equation but that can be referenced:
\begin{equation}
a^2 + b^2 + c^2 \label{eqn:pythagoras}
\end{equation}
Equation~\ref{eqn:pythagoras} has been known for a long time. Sometimes you might want to align equations:
\begin{align*}
(x+y)^2 - (x-y)^2 &= (x^2 + 2xy + y^2) - (x^2 - 2xy - y^2) \\
{}&= 4xy
\end{align*}
or align them with numbers to be referenced:
\begin{align}
(x+y)^2 - (x-y)^2 &= (x^2 + 2xy + y^2) - (x^2 - 2xy - y^2) \label{eqn:1}\\
{}&= 4xy \label{eqn:2}
\end{align}
or where only some of them can be easily referenced:
\begin{align}
\nonumber (x+y)^2 - (x-y)^2 &= (x^2 + 2xy + y^2) - (x^2 - 2xy - y^2) \\
{}&= 4xy \label{eqn:3}
\end{align}

Here are some other math goodies: $\binom{n}{k}$, $\frac{n}{k}$, $x\gg y$, $x\ll y$, $x\lesssim y$, $x\gtrsim y$, $\inprod{x,y}$, $x\in\R^n$, $x\in S$, $S\ni x$, $x\notin S$, $x\neq y$, $x\oplus y$, $x\pm y$, $\sqrt{x}$, $x\equiv y$, $S\cap T$, $\S\cup T$, $S\subset T$, $S\subseteq T$, $S\supset T$, $\gamma$,  $\Gamma$, $\R$, $\mathbb{F}$
$$\sum_{i=1}^n i,\ \sum_{\substack{i=1\\i\ odd}}^n i, \prod_{i=1}^n i$$

\section{Main Section}

We begin by describing the problem \ldots.
Make sure to use sections and subsections.

\subsection{Blah blah blah}
Here is a subsection.

\subsubsection{Blah blah blah}
Here is a subsubsection. You can use these as well.

\subsection{Using Boldface}
Make sure to use \textbf{lots} of {\bf boldface}.

\paragraph{Question:}
How would you use boldface?

\paragraph{Example:}
This is an example showing how to use boldface to 
help organize your lectures.


\paragraph{Some Formatting.}
Here is some formatting that you can use in your notes:
\begin{itemize}
\item {\em Item One} -- This is the first item.
\item {\em Item Two} -- This is the second item.
\item \dots and here are other items.
\end{itemize}

If you need to number things, you can use this style:
\begin{enumerate}
\item {\em Item One} -- Again, this is the first item.
\item {\em Item Two} -- Again, this is the second item.
\item \dots and here are other items.
\end{enumerate}

\paragraph{Bibliography.}
Please give real bibliographical citations for the papers that we
mention in class. See below for how to include a bibliography section.
If you use BibTeX, integrate the .bbl file into your .tex
source. You should reference papers like this: ``The tug of war sketch
originates in a paper by Alon, Matias and Szegedy \cite{AlonMS99}.''
In general, the name of the authors should appear in text at most once 
(for the first citation); further citations look like: ``Our proof follows 
that of \cite{AlonMS99}''.

Take a look at previous lectures (TeX files are available) to see the
details. A excellent source for bibliographical citations is
DBLP. Just Google DBLP and an author's name.


\bibliographystyle{alpha}

\begin{thebibliography}{42}

\bibitem{AlonMS99}
Noga~Alon, Yossi~Matias, Mario~Szegedy.
\newblock The Space Complexity of Approximating the Frequency Moments.
\newblock {\em J. Comput. Syst. Sci.}, 58(1):137--147, 1999.

\end{thebibliography}

\end{document}